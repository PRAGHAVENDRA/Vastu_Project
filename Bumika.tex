\chapter{भूमिका}

अस्मिन् संसारे सर्वोऽपि प्राणी सुखेन जीवितुम् इच्छति । सुखमयजीवनार्थं समीचीने प्रदेशे सुनिर्मितं सुन्दरं गृहम् अत्यावश्यकम् । अत एव प्राचीनकाले भारतदेशे गृहग्रामादिनिर्माणोपायाः चिन्तिताः सन्ति । चिन्तनस्य तस्य वास्तुशास्त्रम् इति नाम्ना व्यवहारः परिदृश्यते । प्रायः भारतीयविद्याः वेदमूलकाः स्मृतिपुराणादिमूलकाः वा भवन्ति । वास्तुविद्यापि तथैव । अथर्ववेदे वास्तुसूत्रोपनिषदिति कश्चन भागः वर्तते । तत्र संक्षेपेण वास्तुविचाराः उपलभ्यन्ते । तथा वेदाङ्गेषु शुल्बसूत्रेषु यज्ञशालायाः, यज्ञस्थण्डिलादीनां निर्माणं विशिष्य चिन्तितम् अस्ति । एवं पुराणेष्वपि वास्तुविचाराः परिदृश्यन्ते । तत्रापि अग्निपुराणे विशिष्य केचन अध्यायाः वास्तुप्रतिपादनार्थमेव रचिताः सन्ति । इत्थं वास्तुविद्या अत्यन्तं प्राचीना इत्यत्र अनेकानि प्रमाणानि उपलभ्यन्ते । 

\section{वास्तुशास्त्रसम्बद्धाः ग्रन्थाः}

इदानीन्तनकाले वास्तुशास्त्रप्रतिपादकाः अनेके ग्रन्थाः समुपलभ्यन्ते । तत्रापि केचन ग्रन्थाः उत्तरभारते गृहादिनिर्माणोपयुक्तान् अंशान् प्रतिपादयन्ति, केचन दक्षिणभारते । देशभेदेन तत्रत्यपरिसरः, जीवनशैली, उपलभ्यमानाः सामग्र्यादयः भिद्यन्ते इत्यतः वास्तुविद्यापि विभिन्ना इति प्रतिभाति । प्रसिद्धाः केचन {\b\bfseries वास्तुग्रन्थाः -}
\begin{itemize}
\item समराङ्गणसूत्रधारः  (भोजराजप्रणीतः)
\item मनुष्यालयचन्द्रिका (केरलदेशीयेन नीलकण्ठेन प्रणीतः)
\item वास्तुविद्या 
\item शिल्परत्नम्
\item मयमतम् (मयमुनिविरचितम्)
\item मानसारः (मानमुनिविरचितः)
\item काश्यपीयम् (कश्यपमुनिः)
\item विश्वकर्मीयम्
\item कुमारतन्त्रम्
\item मार्कण्डेयवास्तुशास्त्रम्
\item क्रियासारः
\item अपराजितपृच्छा
\item क्षीरार्णवः (विश्वकर्मप्रणीतः)
\item विष्णुसंहिता  इत्यादयः
\end{itemize}

\section{मनुष्यालयचन्द्रिका}

एतेषु वास्तुग्रन्थेषु मनुष्यालयचन्द्रिकानामकः ग्रन्थः अत्यन्तं विशिष्टं स्थानं भजते । मनुष्याणां निवासस्य स्थानं मनुष्यालयः इति उच्यते । यथा देवालयः वैद्यालयः इत्यादिः । चन्द्रिका शब्देन चन्द्रस्य किरणः इत्यर्थः लभ्यते । मनुष्याणां निवासस्थानस्य रचनादिकमत्र प्रतिपाद्यते इत्यतः अस्य ग्रन्थस्य मनुष्यालयचन्द्रिका इति नाम । केरलदेशीयेन नीलकण्ठमूसतनाम्ना कविना अयं ग्रन्थः विरचितः । ग्रन्थस्यास्य उपलब्धिः उपयोगश्च दक्षिणभारते अधिकतया भवति । तस्मात् ग्रन्थकारः दक्षिणभारते प्रसिद्धः वास्तुविज्ञानी आसीदिति ज्ञायते । किञ्च ग्रन्थादौ 'श्रीमत्कुण्डपुरे विराजति परक्रोडे च तेजः परम् । नावानाम्नि च धाम्नि यच्च नितरां मल्लीविहारालये ॥'इति श्लोके श्रीमत्कुण्डपुरस्य तथा नावानामकधाम्नः कथनेन ग्रन्थकारः केरलदेशीयः इति ज्ञायते । यतः श्रीमत्कुण्डपुरः एव इदानां तिरुनाना इति नाम्ना व्यवह्रियते । ग्रन्थस्यास्य संस्कृतभाषया रचिताः व्याख्याः यद्यपि नोपलभ्यन्ते तथापि आङ्ग्लभाषायां हिन्दीभाषायां तथा मलयाळकन्नडभाषासु व्याख्यानानि रचितानि सन्ति । तत्रापि मलयाळभाषया रचितं व्याख्यानं विद्यमानेषु प्राचीनतमं वर्तते । वास्तुग्रन्थेषु मनुष्यालयनिर्मातॄणाम् सौकर्यार्थं भाषान्तरानुवादः एव अधिकतया दृश्यते ।

मनुष्यालयचन्द्रिकाग्रन्थे वास्तुविद्यायाः सारः एव कविना सम्पादितः । अत एव आदौ 'मयमतयुगलं प्रयोगमञ्जर्यपि च निबन्धभास्करीययुग्मम् । मनुमतगुरुदेवपद्धतिश्रीहरियजनादिमहागमा जयन्ति ॥'इति कथयति कविः । तस्य अभिप्रायस्तु मयेन लिखितं ग्रन्थद्वयं तथा च प्रयोगमञ्जरी, निबन्धभास्करीयम् , मनुमतमित्यादिषु ग्रन्थेषु प्रतिपादितानां विषयाणां सारः अत्र सङ्गृह्यते इति । ग्रन्थेऽस्मिन् सप्त(7) अध्यायाः सन्ति । तेषु षट्चत्वारिंशदधिकद्विशतसङ्ख्याकाः (246) श्लोकाः निबद्धाः सन्ति । ग्रन्थेऽस्मिन् विशिष्य मनुष्याणां निवासगृहस्य निर्माणम् , तथा तदुपयोगितया कूपादिनिर्माणम्, वृक्षाद्यारोपणम्, परिमाणभेदाः, भूपरीक्षा, दिङ्निर्णयः, स्तम्भविन्यासः इत्यादिविचाराः प्रदर्शिताः सन्ति ।

\section{ग्रन्थस्य भाषा}

उपलभ्यमानेषु वास्तुग्रन्थेषु अयं ग्रन्थः भाषादृष्ट्यापि विशिष्टः वर्तते । ग्रन्थेऽस्मिन् अपाणिनीयाः शब्दप्रयोगाः प्रायः न लभ्यन्ते । तथा समस्तपदानां बाहुल्यं विद्यते । श्लोकात्मकः अयं ग्रन्थः । तस्मात् छन्दसः अपि उपलब्धिः अत्र भवति । अनुष्टुबादिभिः अनेकैः छन्दोभिः श्लोकाः निबद्धाः सन्ति । ग्रन्थेस्मिन् उपलभ्यमानानि छन्दांसि - अनुष्टुप् , उपजातिः, शार्दूलविक्रीडितम् , वसन्ततिलका, पुष्पिताग्रा,  इत्यादीनि । वास्तुविद्यायाः प्रतिपादनार्थं क्रियापदानाम् आवश्यकता विद्यते एव । अतः एकैकस्मिन् श्लोके एकं वा तिङन्तपदं दृश्यते । किञ्च प्रायः विधिलिङ्-लोट्लकाराणां तथा तव्यत्प्रत्ययान्तानां प्रयोगः बहुलतया दृश्यते । यथा - "जानीयात् स्थापनार्हं स्थपतिमथगुणैः"\footnote{१.१३} इत्यस्मिन् श्लोके गृहदेवालयादिनिर्माणं यः जानाति तथा च गृहादिनिर्माणकुशलता यस्मिन् वर्तते सः स्थपतिः इत्युच्यते इति कथनार्थं स्थपतिं जानीयात् इत्युक्तम् । जानीयादिति ज्ञा(अवबोधने)धातोः विधिलिङि प्रथमपुरुषैकवचनान्तं रूपम् । तथा "अन्तर्भागं त्रिभागं नयतु गतदिनाङ्कं तदेवेह सूक्ष्मम्"\footnote{२.३} इत्यस्मिन् श्लोके दिशः निर्णयसन्दर्भे भागत्रयादिगणितं कृत्वा प्राचीदिक् सम्पादनीया इत्यर्थे नयतु इति शब्दः दृश्यते । नयतु इति नी(प्रापणे)धातोः लोट्लकारे प्रथमपुरुषस्य एकवचनान्तं रूपम् ।  क्वचित् लट्लकारः अपि प्रयुक्तः । यथा "प्राहुर्गेहचतुष्कपादुकबहिर्भागं बुधाः प्राङ्गणम्"\footnote{५.१} इत्यस्मिन् श्लोके ज्ञानिनः प्राङ्गणमिति कथयन्ति इति बोधनार्थं प्राहुः इति तिङन्तं प्रयुक्तम् । प्राहुरिति प्रोपसर्गपूर्वकब्रूञ्(व्यक्तायां वाचि)धातोः लट्लकारे प्रथमपुरुषैकवचनान्तं रूपम् । क्वचित् विधिबोधनार्थं तव्यत्प्रत्ययान्तं रूपमपि प्रयुक्तं दृश्यते । यथा "बालत्वादिविशेषमात्रमवगन्तव्यं फलैः शेषतः"\footnote{३.३८} इति आयव्ययादिगणनासन्दर्भे या सङ्ख्या अवशिष्यते सा बाल्यादिवयः बोधयति ।तेन शुभाशुभादिकं ज्ञातुं शक्यते इति अत्र उच्यते । तत्र एवं गणनां कृत्वा ज्ञातव्यमित्यर्थे अवगन्तव्यमिति तव्यत्प्रत्ययान्तं रूपं प्रयुक्तम् ।

एवम् अस्मिन् ग्रन्थे समस्तपदानि बाहुल्येन उपलभ्यन्ते । यथा "करकिष्क्वरत्निभुजदोर्मुष्ट्यादिसञ्ज्ञम्"\footnote{३.१} इति श्रूयते । अस्य विग्रहस्तु एवमस्ति - करः च किष्कुः च अरत्निः च भुजः च दोः च मुष्टिः च = करकिष्क्वरत्निभुजदोर्मुष्टयः (इतरेतरद्वन्द्वः), करकिष्क्वरत्निभुजदोर्मुष्टयः आदिः येषां ते = करकिष्क्वरत्निभुजदोर्मुष्ट्यादयः (बहुव्रीहिः) , करकिष्क्वरत्निभुजदोर्मुष्ट्यादयः सञ्ज्ञा यस्य तत् = करकिष्क्वरत्निभुजदोर्मुष्ट्यादिसञ्ज्ञम् (बहुव्रीहिः) । तथा "मत्स्यानेकपकूर्मपृष्ठकपिलावक्त्रोपमा"\footnote{१.१८} – मत्स्यः च अनेकपः च कूर्मः च = मत्स्यानेकपकूर्माः, मत्स्यानेकपकूर्माणां पृष्ठम् = मत्स्यानेकपकूर्मपृष्ठम्, कपिलायाः वक्त्रम् = कपिलावक्त्रम्,  मत्स्यानेकपकूर्मपृष्ठं  च कपिलावक्त्रं च = मत्स्यानेकपकूर्मपृष्ठकपिलावक्त्रे, मत्स्यानेकपकूर्मपृष्ठकपिलावक्त्राभ्याम् उपमा यस्याः सा = मत्स्यानेकपकूर्मपृष्ठकपिलावक्त्रोपमा  ।

मनुष्यालयचन्द्रिकाकारः पुस्तकेऽस्मिन् विशिष्टानां तद्धितान्तानां प्रयोगमपि बहुलतया कृतवान् । यथा - दिशः बोधनार्थं 'वारुणी'इत्यादिपदानि प्रयुक्तानि । वरुणशब्दस्य अण्प्रत्ययान्तं रूपमिदं दिग्बोधकमित्यतः स्त्रीप्रत्ययान्तमपि वर्तते । एवं सङ्ख्याबोधनार्थम् उपायान्तरम् अनुसृतम् । यथा "मातङ्गभास्करनृपाङ्गुलमात्रतुङ्गम्"\footnote{५.१} इत्यत्र मातङ्गशब्दः  अष्टसङ्ख्याम्, भास्करशब्दः द्वादशसङ्ख्याम्, नृपशब्दः षोडशसङ्ख्यां च द्योतयति ।

ग्रन्थेऽस्मिन् गृहादिनिर्माणविचाराः सन्ति इत्यतः परिमाणबोधकपदानि अधिकतया उपलभ्यन्ते । अतः परिमाणबोधकपदानां तथा परिमाणभेदानां प्रदर्शनं  श्लोके सङ्गृहीतम् अस्ति । 

{\b\bfseries परिमाणम्}\hfil -\hfil {\b\bfseries परिमाणस्य नाम}
\begin{itemize}
\item अष्टौ तिलोदराणि\hfil -\hfil एकः यवोदरः
\item अष्टौ यवोदराः\hfil -\hfil एकम् मात्राङ्गुलम्
\item द्वादश मात्राङ्गुलानि\hfil -\hfil एकः वितस्तिः
\item द्वौ वितस्ती (चत्वारिंशत् अङ्गुलानि)\hfil -\hfil करः,किष्कुः,अरत्निः,भुजः,दोः
\end{itemize}

\section{दण्डभेदाः तेषां नामानि च}

{\b\bfseries दण्डस्य नाम}\hfil -\hfil {\b\bfseries दण्डपरिमाणम्}
\begin{itemize}
\item किष्कुः\hfil -\hfil २४-अङ्गुलानि
\item प्राजापत्यम्\hfil -\hfil २५-अङ्गुलानि
\item धनुर्मुष्टिः\hfil -\hfil २६-अङ्गुलानि
\item धनुर्ग्रहम्\hfil -\hfil २७-अङ्गुलानि
\item प्राच्यम्\hfil -\hfil २८-अङ्गुलानि
\item वैदेहः\hfil -\hfil २९-अङ्गुलानि
\item वैपुल्यम्\hfil -\hfil ३०-अङ्गुलानि
\item प्रकीर्णः\hfil -\hfil ३१-अङ्गुलानि
\end{itemize}

\section{ग्रन्थेऽस्मिन् प्रतिपादिताः विषयाः}

ग्रन्थकारः आदौ नृसिंहयादवयोः नमस्कारं कुर्वन् मङ्गलम् आचरति । ततः ग्रन्थरचनायाः उद्देश्यं तथा विषयस्य निर्देशनं च करोति । तदनन्तरं गृहनिर्माणार्थम् आदौ स्थपत्यादिभ्यः दायित्वप्रदानम्, गृहनिर्माणकाले अवधातव्याः अंशाः, भूपरीक्षा, गृहाणि परितः आरोपणीयाः वृक्षाः इत्यादिकं प्रथमाध्याये सङ्गृहीतवान् । {\b\bfseries द्वितीयाध्याये} प्राच्यादीनां दिशां निर्णयः सूर्यगत्यनुगुणं कथं सम्पादनीयः इति प्रधानतया प्रदर्शितवान् । तथा अत्र वास्तुदेवतासम्बद्धविचाराः प्रतिपादिताः । {\b\bfseries तृतीयाध्याये} तु आदौ परिमाणभेदाः प्रदर्शिताः । तत्रापि तिलोदरादिकिष्कुपर्यन्तं सूक्ष्मपरिमाणभेदाः तथा किष्क्वादिप्रकीर्णपर्यन्तं दण्डभेदाः उक्ताः । गृह-ग्राम-वीथ्यादिनिर्माणे कीदृशः परिमाणः मूलभूततया धर्तव्यः इत्यादिकं सविस्तरेण प्रतिपादितम् । किञ्च आयनिर्णयः अत्र प्रदर्शितः । तदर्थं गणितप्रक्रिया अत्र दृश्यते । {\b\bfseries चतुर्थाध्याये} तु प्रकोष्ठादिनिर्माणविषये कथितमस्ति । तथा गृहस्य विस्तारः, तदनुसारं प्रकोष्ठादीनां रचना इत्यादिकम् उक्तम् । {\b\bfseries पञ्चमे} तु आयस्य(Foundation) रचनाभेदाः, तुलसीपीठस्य रचना, गृहाङ्गणस्य रचना, गृहान्तरङ्गणस्य रचना, एवं स्तम्भस्य(Pillar) रचना, स्तम्भानां संयोजनमित्यादिकम् उक्तमस्ति । {\b\bfseries षष्ठे} अध्याये गृहस्य उपरिभागे गृहाच्छादनार्थं लुपालम्बनिर्माणक्रमः प्रदर्शितः । {\b\bfseries सप्तमे} तु द्वारकवाटादीनां रचना, प्रकोष्ठादीनाम् उपयोगः, भोजनगृहादिकम् , कूपादिनिर्माणम्, राज्ञां गृहाणि, प्राकारादिरचना इत्यादिविचाराः उपलभ्यन्ते । इत्थं सप्तसु अध्यायेषु गृहतत्सम्बद्धविचाराः अत्र समुपलभ्यन्ते।
