\chapter{भूपरीक्षापरिग्रहः}	

{मङ्गलश्लोकः –} ग्रन्थकारः विघ्नानां निवारणार्थं ग्रन्थारम्भवेलायां इष्टदेवतायाः स्तुतिरूपं मङ्गलम् आचरति ।

\begin{nscenter}
{\b\bfseries नृसिंहयादवाकारतेजोद्वितयमद्वयम् ।\\[-1mm]
राजते नितरां  राजराजमङ्गलधामनि ॥१॥}
\end{nscenter}

{पदच्छेदः –} नृसिंह-यादवाकार-तेजो-द्वितयम्, अद्वयम्, राजते, नितराम्, राजराजमङ्गलधामनि ।

{पदविश्लेषणम् –}
\usepackage[table]{xcolor}
%\rowcolors{1}{green!25}{green!75}\cellcolor{gray}
\begin{tabular}{|c|c|c|}
\hline
\rowcolor{green!25} {पदम्} & {पदपरिचयः} & {भावार्थः} \hline
नृसिंहयादवाकारतेजोद्वितयम् & अकार.नपुं.प्रथमा.एक. & नृसिंहरूपेण श्रीकृष्णरूपेण च विद्यमानं तेजः \hline 
अद्वयम् &  अकार.नपुं.प्रथमा.एक.  & अभिन्नम् \hline
राजते & राजृ(दीप्तौ)धातुः, लट्.प्रथम.एक. & प्रकाशवत् शोभते \hline 
नितराम् & अव्ययम् & अत्यन्तम् \hline
राजराजमङ्गलधामनि & (धामन्)नकार.नपुं.सप्तमी.एक. (संज्ञावाचकं पदम्) & ग्रामस्य नाम \hline 
\end{tabular}

{अन्वयः -} राजराजमङ्गलधामनि नृसिंह-यादवाकारतेजो-द्वितयम् अद्वयं नितरां राजते ।

{भावार्थः –} राजराजमङ्गलधाम इति मन्दिरं केरलराज्ये मलप्पुरजिल्लायां वर्तते । तत्र नृसिंहरूपेण श्रीकृष्णरूपेण च अभिन्नतया भगवान् विराजते ।

{व्याकरणांशाः –}\\
नृसिंहयादवाकारतेजोद्वितयम् – नृसिंहयादवाकारतेजः + द्वितयम् (विसर्गसन्धिः)\\ नृसिंहः च यादवः च = नृसिंहयादवौ (इतरेतरद्वन्द्वः) । नृसिंहयादवयोः आकारः यस्य तत् नृसिंहयादवाकारम् (बहुव्रीहिः) । नृसिंहयादवाकारं  च तत् तेजः च = नृसिंहयादवाकारतेजः(कर्मधारयः) । नृसिंहयादवाकारतेजसां द्वितयम् = नृसिंहयादवाकारतेजोद्वितयम् (षष्ठीतत्पुरुषः) ।
अद्वयम् – न द्वयम् (नञ्-तत्पुरुषः) ।
नितराम् – नि + तरप् ।

\begin{nscenter}
{\b\bfseries श्रीमत्कुण्डपुरे विराजति परक्रोडे च तेजः परम्\\[-1mm]
नावानाम्नि च धाम्नि यच्च नितरां मल्लीविहारालये ।\\[-1mm]
अश्वत्थाख्यनिकेतनेऽपि च पुरे श्रीकेरलाधीश्वरे[-1mm]
सम्भूयैतदुरुप्रकाशविषये चित्ते ममोज्जृम्भताम् ॥२॥}
\end{nscenter}

{पदच्छेदः –} श्रीमत्-कुण्डपुरे, विराजति, पर-क्रोडे, च, तेजः, परम् , नावा-नाम्नि, च, धाम्नि, यत् , च, नितराम् , मल्ली-विहार-आलये, अश्वत्थाख्य-निकेतने, अपि, च, पुरे, श्रीकेरल-अधीश्वरे, सम्भूय, एतद्-उरु-प्रकाश-विषये, चित्ते, मम, उज्जृम्भताम् ।

{पदविश्लेषणम् –}
\usepackage[table]{xcolor}
%\rowcolors{1}{green!25}{green!75}\cellcolor{gray}
\begin{tabular}{|c|c|c|}
\hline
\rowcolor{green!25} {पदम्} & {पदपरिचयः} & {भावार्थः} \hline
श्रीमत्कुण्डपुरे & अकार.पुं.सप्तमी.एक. (संज्ञावाचकं पदम्) & कुण्डपुरनामके ग्रामे \hline 
विराजति & वि-राजृ(दीप्तौ)धातुः, लट्.प्रथम.एक. & शोभते \hline 
परक्रोडे & अकार.पुं.सप्तमी.एक. (संज्ञावाचकं पदम्) & प्रक्रोडनामके ग्रामे \hline 
च & अव्ययम् & समुच्चयः \hline 
तेजः & (तेजस्)सकार.नपुं.प्रथमा.एक. & 	प्रकाशः \hline 
परम् & अकार.नपुं.प्रथमा.एक. & श्रेष्ठम् \hline 
नावानाम्नि & (नामन्)नकार.नपुं.सप्तमी.एक. & नावानामके \hline 
धाम्नि & (धामन्)नकार.नपुं.सप्तमी.एक. & प्रदेशे \hline 
यत् & (यद्)दकार.नपुं.प्रथमा.एक. & यत् परं तेजः \hline 
नितराम् & अव्ययम् & अत्यन्तम् \hline
मल्लीविहारालये & अकार.पुं.सप्तमी.एक. & 	मल्लिविहारप्रदेशे देवालये \hline 
अश्वत्थाख्यनिकेतने & अकार.नपुं.सप्तमी.एक. & अश्वत्थप्रदेशस्थिते देवालये \hline 
अपि & अव्ययम् &  समुच्चयः \hline 
पुरे & अकार.पुं.सप्तमी.एक. & प्रदेशे \hline 
श्रीकेरलाधीश्वरेे & अकार.पुं.सप्तमी.एक. & 	केरलाधीश्वरनामके देवालये \hline 
सम्भूय & ल्यप्-प्रत्ययान्तम् अव्ययम् & मिलित्वा \hline 
एतदुरुप्रकाशविषये & अकार.पुं.सप्तमी.एक. & वास्तोः वाचारस्य प्रतिपादनसन्दर्भे \hline 
चित्ते & अकार.नपुं.सप्तमी.एक. & मनसि \hline 
मम & (अस्मद्)दकार.त्रिषु लिङ्गेषु समानः.षष्ठी.एक. & मम \hline 
उज्जृम्भताम् & उत्-जृभि(गात्रविनामे)धातुः, विधिलिङ्.प्रथम.एक. & विजृम्भणं भवतु \hline
\end{tabular}

{अन्वयः -} श्रीमत्कुण्डपुरे, परक्रोडे च, नावानाम्नि धाम्नि च, मल्लीविहारालये, अश्वत्थाख्यनिकेतने च, अपि च श्रीकेरलाधीश्वरे पुरे यत् परं तेजः नितरां विराजति (तत्) सम्भूय एतदुरुप्रकाशविषये मम चित्ते उज्जृम्भताम् ।

{भावार्थः – कुण्डपुरे परक्रोडे नावायां मल्लीविहारे अश्वत्थे तथा केरलेषु ये ये देवालयाः सन्ति तेषु यः उत्कृष्टः प्रकाश्यमानः देवः विराजते सः वास्तुसम्बनद्धानां विषयाणां प्रतिपादनसन्दर्भे मम मनसि भवतु । तदानीं तस्य स्मरणेन मम विघ्नाः दूरं गमिष्यन्ति ।}

{व्याकरणांशाः –}\\
श्रीमत्कुण्डपुरे – श्रीमत् + कुण्डपुरे । श्रीः अस्य अस्तीति श्रीमान् (मतुप्-प्रत्ययान्तम्), श्रीमान् च सः कुण्डपुरः = श्रीमत्कुण्डपुरः (कर्मधारयः), तस्मिन् ।
नावानाम्नि – नावा इति नाम यस्य तत् = नावानाम (बहुव्रीहिः), तस्मिन् ।
यच्च – यत् + च (श्चुत्वसन्धिः)
मल्लीविहारालये – मल्लीविहार + आलयः = मल्लीविहारालयः (सवर्णदीर्घसन्धिः) । मल्लीविहारः च असौ आलयः च = मल्लीविहारालयः (कर्मधारयः), तस्मिन् ।
अश्वत्थाख्यनिकेतनेऽपि – अश्वत्थाख्यनिकेतने + अपि (पूर्वरूपसन्धिः) । अश्वत्थम् आख्या यस्य तत् = अश्वत्थाख्यम् (बहुव्रीहिः), अश्वत्थाख्यं च तत् निकेतनं च = अश्वत्थाख्यनिकेतनम् (कर्मधारयः), तस्मिन् ।
श्रीकेरलाधीश्वरे – श्रीकेरल + अधीश्वरः (सवर्णदीर्घसन्धिः), श्रीकेरलाधि + ईश्वरः (सवर्णदीर्घसन्धिः) । अधिकः ईश्वरः = अधीश्वरः (प्रादितत्पुरुषः), केरलानाम् अधीश्वरः = केरलाधीश्वरः (षष्ठीतत्पुरुषः) । श्रीसहितः केरलाधीश्वरः = केरलाधीश्वरः (मध्यमपदलोपिसमासः), तस्मिन् ।
सम्भूयैतदुरुप्रकाशविषये – सम्भूय + एतत् (वृद्धिसन्धिः), सम्भूयैतत् + उरुप्रकाशविषये (जश्त्वसन्धिः) । उरुः प्रकाशः = उरुप्रकाशः (कर्मधारयः), उरुप्रकाशस्य विषयः = उरुप्रकाशयविषयः (षष्ठीतत्पुरुषः), एषः उरुप्रकाशविषयः = एतदुरुप्रकाशविषयः (कर्मधारयः), तस्मिन् ।
ममोज्जृम्भताम् – मम + उज्जृम्भताम् (गुणसन्धिः) ।
